\documentclass[12pt, a4paper]{article}
\usepackage{blindtext, titlesec, amsthm, thmtools, amsmath, amsfonts, scalerel, amssymb, graphicx, titlesec, xcolor, multicol, hyperref}
% \usepackage{mathtools}
\usepackage{bm}
\usepackage{etoolbox}
\AtBeginEnvironment{bmatrix}{\setlength{\arraycolsep}{2pt}}
\usepackage[utf8]{inputenc}
\hypersetup{colorlinks,linkcolor={red!40!black},citecolor={blue!50!black},urlcolor={blue!80!black}}
\newtheorem{theorem}{Theorema}[subsection]
\newtheorem{lemma}[theorem]{Lemma}
\newtheorem{corollary}[theorem]{Corollarium}
\newtheorem{hypothesis}{Coniectura}
\theoremstyle{definition}
\newtheorem{definition}{Definitio}[section]
\theoremstyle{remark}
\newtheorem{remark}{Observatio}[section]
\newtheorem{example}{Exampli Gratia}[section]
\renewcommand\qedsymbol{Q.E.D.}
\title{Math For Computer Graphics}
\author{Harry Han}
\date{\today}
\begin{document}
\maketitle
%\tableofcontents
\section{Linear Transformations}

A point (or vertex) in three dimensional space can be represented by the following vector: $\bm{v}=[x_0, y_0 ,z_0, 1]^{T}$. The extra dimension is solely for the convenience of matrix manipulation.

\subsection{Translation, Rotation}

Translation of a vertex $\bm{v}=[x_0, y_0,z_0, 1]^T$ in $x$ direction for $t_x$, $y$ direction for $t_y$, and $z$ direction for $t_z$ is a linear transfomation, whose corresponding matrix is $T$:
\begin{equation}\label{eq:translation}
	T\bm{v} = 
	\begin{bmatrix}
	1 & 0 & 0 & x_t\\
	0 & 1 & 0 & y_t\\
	0 & 0 & 1 & z_t\\
	0 & 0 & 0 & 1
	\end{bmatrix} 
	\begin{bmatrix}
	x_0\\
	y_0\\
	z_0\\
	1
	\end{bmatrix}
	= [x_0 + x_t, y_0 + y_t, z_0 + z_t, 1]^T
\end{equation}

The matrices $R_x, R_y, R_z$ that correspond to the rotations of a vertex $\bm{v}=[x_0, y_0, z_0, 1]^T$ respect to $x, y, z$ for $\theta$ degrees are:
\begingroup
\fontsize{10}{12}\selectfont
\begin{equation}\label{eq:rotation}
	R_z = 
	\begin{bmatrix}
	\cos{\theta} & -\sin{\theta} & 0 & 0\\
	\sin{\theta} & \cos{\theta} & 0 & 0\\
	0 & 0 & 1 & 0\\
	0 & 0 & 0 & 1
	\end{bmatrix}, 
	R_x =
	\begin{bmatrix}
	1 & 0 & 0 & 0\\
	0 & \cos{\theta} & -\sin{\theta} & 0\\
	0 & \sin{\theta} & \cos{\theta} & 0\\
	0 & 0 & 0 & 1
	\end{bmatrix}, 
	R_y =
	\begin{bmatrix}
	\cos{\theta} & 0 & \sin{\theta} & 0\\
	0 & 1 & 0 & 0\\
	-\sin{\theta} & 0 & \cos{\theta} & 0\\
	0 & 0 & 0 & 1
	\end{bmatrix}
\end{equation}
\endgroup

The Scaling matrix $S$ is:
\begin{equation}\label{eq:scaling}
	S = 
	\begin{bmatrix}
	s_x & 0 & 0 & 0\\
	0 & s_y & 0 & 0\\
	0 & 0 & s_z & 0\\
	0 & 0 & 0 & 1
	\end{bmatrix}
\end{equation}

Recall matrix under mutiplication is not an abelien group, i.e., matrix multiplication is associative, although not commutative.
To tranform a vertex $\bm{v}$ with scaling, rotation, and translation, the order of the matrix is important:
\begin{equation}\label{eq:combinedTransformation}
TR_xR_yR_zS\bm{v}
\end{equation}

OpenGL will normalize all the vertices before rendering to screen, which means if the setted screen is not a square, the final rendering will be distorted.

\subsection{Projection}

\end{document}


\documentclass[12pt, a4paper]{article}
\usepackage{blindtext, titlesec, amsthm, thmtools, amsmath, amsfonts, scalerel, amssymb, graphicx, titlesec, xcolor, multicol, mathtools}
\usepackage{bm}
%\usepackage{hyperref}
\usepackage[utf8]{inputenc}
%\hypersetup{colorlinks,linkcolor={red!40!black},citecolor={blue!50!black},urlcolor={blue!80!black}}
\newtheorem{theorem}{Theorema}[subsection]
\newtheorem{lemma}[theorem]{Lemma}
\newtheorem{corollary}[theorem]{Corollarium}
\newtheorem{hypothesis}{Coniectura}
\theoremstyle{definition}
\newtheorem{definition}{Definitio}[section]
\theoremstyle{remark}
\newtheorem{remark}{Observatio}[section]
\newtheorem{example}{Exampli Gratia}[section]
\renewcommand\qedsymbol{Q.E.D.}
\title{On Linear Regression}
\author{Harry Han}
\date{\today}
\begin{document}
\maketitle

\section{Notation}
\begin{enumerate}
	\item $\bm{x}$ is used to denote a vector (or data set)
	\item $\bar{\bm{x}}$ denotes the mean of the vector (data set). 
\end{enumerate}

\section{The Simplest: $y=ax+b$}
\begin{definition}[$R^2$]
	
\end{definition}

For a data set $\bm{x}=[x_0, x_1, x_2, \cdots, x_{n-1}]^T$ and $\bm{y}=[y_0, y_1, y_2, \cdots, y_{n-1}]^T$ of the same dimension, 
there exists a linear regression $y=ax+b$ such that the $R^2$ value is minimized. 

The promised regression is:
\begin{equation}
\label{eq:SimplestRegression}
a = \frac{\bm{x}\cdot \bm{y}-n\bar{\bm{x}}\bar{\bm{y}}}{(\bm{x}\cdot \bm{x})-n\bar{\bm{x}}^2};\ 
b = \frac{(\bm{x}\cdot \bm{x})\bar{\bm{y}}-\bar{\bm{x}}(\bm{x}\cdot \bm{y})}{(\bm{x}\cdot \bm{x})-n\bar{\bm{x}}^2}
\end{equation}

notice: 
$$b=\bar{\bm{y}}-a\bar{\bm{x}}$$

\end{document}

